\documentclass[A4,12pt]{scrartcl}% ===> this file was generated automatically by noweave --- better not edit it

%% prelude.tex
%% https://tex.stackexchange.com/questions/36609/formatting-section-titles
%% KOMA needs it
\let\sf\sffamily
\let\bf\bfseries
\let\tt\ttfamily
\let\it\itshape
\let\rm\rmfamily

\usepackage[nodayofweek]{datetime}
\usepackage{boolexpr,pdftexcmds,trace}
\usepackage{multicol}

%% KOMA for headers :
\usepackage{scrlayer-scrpage}
\pagestyle{scrheadings}
\automark[section]{section}
\cfoot[\pagemark]{\pagemark}
%% for date/time
% \usepackage{scrdate}
% \usepackage{scrtime}

\usepackage{xifthen}% provides \isempty test

%% package for noweb
\usepackage{common/newnoweb}
\noweboptions{smallcode,longchunks,longxref}

%% fonts and encoding
\usepackage[T1]{fontenc}
\usepackage[condensed,sfdefault]{universalis}
\usepackage{CormorantGaramond}
\usepackage[zerostyle=d]{newtxtt}
\renewcommand{\familydefault}{\rmdefault}

\usepackage[utf8]{inputenc}
% Babel for French is known to break KOMA scripts
% \usepackage[french]{babel}

%% packages needed for section coloring
\usepackage[table]{xcolor}
\usepackage{tikz}

%% hyperlinks initialisation
\definecolor{mygray}{rgb}{0.4,0.4,0.4}
\usepackage[bookmarks,backref=page,linkcolor=mygray]{hyperref} %,colorlinks
\hypersetup{%
  pdfauthor = {Bernard Tatin},
  pdftitle = {},
  pdfsubject = {},
  pdfkeywords = {},
  colorlinks=true,
  linkcolor= mygray,
  citecolor= black,
  pageanchor=true,
  urlcolor = mygray,
  plainpages = false,
  linktocpage
}

\renewcommand{\thesection}{\thepart.\arabic{section}}
% from : https://tex.stackexchange.com/questions/245979/numberwithinequationsubsection-fails-for-subsections-0
%% from https://tex.stackexchange.com/questions/192838/part-heading-style?rq=1
\newcommand{\addtocontenttable}[4]{%
  \refstepcounter{#1}%
  \addcontentsline{toc}{#1}{#2\hspace{#3}#4}%
}

%% parameters :
%   1 -> text style (\Huge\bfseries for part, \Large\bfseries for sections...)
%   2 -> the counter (\thepart, \thesection{arg} ...)
%   3 -> position above baseline (-4cm for the part, -1.5cm for sections)
\newcommand{\boxforcount}[3]{%
  \begin{minipage}{7mm}
    \begin{tabular}{@{}p{7mm}@{}}
      \makebox[7mm]{%
        \cellcolor{orange}\color{white}
        #1
        \strut
        #2
        \rule[#3]{0pt}{4cm}
      }%
    \end{tabular}%
  \end{minipage}
}

% reset section counter at each part
\makeatletter
\@addtoreset{section}{part}

%% for vbox, see https://tex.stackexchange.com/questions/83930/what-are-the-different-kinds-of-boxes-in-latex
\def\@part[#1]#2{%
\cleardoublepage
  \addtocontenttable{part}{\thepart}{1em}{#1}
  \mbox{%
    \boxforcount{\Huge\bfseries}{\thepart}{-4cm}
    \kern-3pt
    \begin{minipage}{140mm}
       \tabular[t]{@{}p{1cm}p{\dimexpr\hsize-2.1cm}@{}}\hline
          & \Huge\bfseries\itshape\rule{0pt}{1.5\ht\strutbox}
          #1
          \endtabular
    \end{minipage}
    } %% mbox
  % \cleardoublepage
 \vskip 100\p@
} %% \def\@part



\long\def\isequal#1#2{\pdf@strcmp{#1}{#2}}

%% colorized sections
\setkomafont{section}{\mysection}
\newcommand{\mysection}[1]{%
  \mbox{%
    \boxforcount{\Large\bfseries}{%
      \switch
        \case{\isequal{\thesection}{.0}}
          {}
        \otherwise
          \thesection{}
      \endswitch
    }{-1.5cm}
    \kern-3pt
    \begin{minipage}{140mm}
      \color{mygray}
       \tabular[t]{@{}p{1cm}p{\dimexpr\hsize-2.1cm}@{}}\hline
          & \Large\bfseries\itshape\rule{0pt}{1.5\ht\strutbox}
          #1
          \endtabular
    \end{minipage}
    }
}

%% colorized subsections
\setkomafont{subsection}{\submysection}
\newcommand{\submysection}[1]{%
    \large\sffamily%
    \color{white}%
    \setlength{\fboxsep}{0cm}%already boxed
    \colorbox{orange!80}{%
        \begin{minipage}{\linewidth}%
            \vspace*{4pt}%Space before
            \hspace*{3pt}
            #1
            \vspace*{4pt}%Space after
        \end{minipage}%
    }}

%% colorized subsubsections
\setkomafont{subsubsection}{\subsubmysection}
\newcommand{\subsubmysection}[1]{%
    \itshape\sffamily%
    \color{white}%
    \setlength{\fboxsep}{0cm}%already boxed
    \colorbox{orange!80}{%
        \begin{minipage}{\linewidth}%
            \vspace*{4pt}%Space before
            \hspace*{3pt}
            #1
            \vspace*{4pt}%Space after
        \end{minipage}%
    }}

\makeatother

%% a better abstract (I hope...)
% french date format
\newdateformat{mydate}{\twodigit{\THEDAY}{/}\twodigit{\THEMONTH}{/}\THEYEAR}

\newcommand{\opennote}[1]{%
  \hfill
  \begin{minipage}
    {0.95\textwidth}
    \rule{\textwidth}{1pt}
    \textbf{#1:\ }
    \footnotesize
}
\newcommand{\closenote}{%
  \normalsize
    \par\noindent
    \rule{\textwidth}{1pt}
  \end{minipage}
}

\newenvironment{note}[1]{%
  \opennote{#1}
}{%
  \closenote
}

\newenvironment{termdefinition}[1]{%
  \opennote{Définition de #1}
}{%
  \closenote
}

\renewenvironment{abstract}[1]{%
  \hfill
  \begin{note}{Résumé}
    #1\footnote{Document crée le \mydate\today{} à \currenttime{}.}
  \end{note}
  \clearpage
}

%% better itemize
\newenvironment{packed_itemize}{
\begin{itemize}
  \setlength{\itemsep}{0pt}
  \setlength{\parskip}{0pt}
  \setlength{\parsep}{0pt}
}{\end{itemize}}



\author{Bernard Tatin}
\date{2017}
\title{À propos des continuations}
\begin{document}

\maketitle
\abstract{Ici, on s'occupe des continuations tout d'abord avec \emph{Scheme} puis, si possible,
avec d'autres langages dont \emph{Standard ML} ou \emph{F\#}. Le fil conducteur provient, sauf
indication contraire, des articles de \emph{Wikipedia} en anglais ou en français qui concernent
ces continuations de la programmation fonctionnelle.\\
\\
Le choix de \texttt{noweb} provient du simple fait que documentation et sources sont conçus
en même temps.}

\tableofcontents

\part{introduction}

\rawpicture{small-loire.png}{la Loire, métaphore de la continuation?}{photo de l'auteur}{r}

\begin{note}{Note}
Ce qui suit provient pour l'essentiel de l'article de Wikipedia, \cite{wikienCPS} (en anglais): \href{https://en.wikipedia.org/wiki/Continuation-passing_style}{Continuation-passing style}.
\end{note}

\section{un premier test et quelques définitions}
Commençons donc par les définitions essentielles:

\begin{termdefinition}{CPS}
Le \emph{continuation-passing style} ou \textbf{CPS} est un style de programmation où le contrôle est passé explicitement sous forme de continuation.
\end{termdefinition}

C'est ce style que nous allons présenter dans les pages qui suivent. En attendant, voyons ce qu'est une continuation :

\begin{termdefinition}{continuation}
Une continuation d'un programme est \emph{la suite des instructions qu'il lui reste à exécuter à un moment précis}\footnote{\emph{Cf.}
\cite{wikifrcontinuations}.}
\end{termdefinition}

Voici un exemple montrant la différence entre le style direct:
\nwfilename{_noweb.nw}\nwbegincode{1}\sublabel{NW48t7P8-1q4Jku-1}\nwmargintag{{\nwtagstyle{}\subpageref{NW48t7P8-1q4Jku-1}}}\moddef{distance direct style~{\nwtagstyle{}\subpageref{NW48t7P8-1q4Jku-1}}}\endmoddef\nwstartdeflinemarkup\nwusesondefline{\\{NW48t7P8-3l3wO2-1}}\nwenddeflinemarkup
(define (\nwlinkedidentc{distance}{NW48t7P8-1q4Jku-1} x y)
 (sqrt (+ (* x x) (* y y))))
\nwindexdefn{\nwixident{distance}}{distance}{NW48t7P8-1q4Jku-1}\eatline
\nwused{\\{NW48t7P8-3l3wO2-1}}\nwidentdefs{\\{{\nwixident{distance}}{distance}}}\nwendcode{}\nwbegindocs{2}\nwdocspar
Et le CPS:
\nwenddocs{}\nwbegincode{3}\sublabel{NW48t7P8-5trcp-1}\nwmargintag{{\nwtagstyle{}\subpageref{NW48t7P8-5trcp-1}}}\moddef{distance CPS~{\nwtagstyle{}\subpageref{NW48t7P8-5trcp-1}}}\endmoddef\nwstartdeflinemarkup\nwusesondefline{\\{NW48t7P8-3l3wO2-1}}\nwenddeflinemarkup
(define (\nwlinkedidentc{distance}{NW48t7P8-1q4Jku-1}\nwlinkedidentc{-cps}{NW48t7P8-5trcp-1} x y k)
 (*-cps x x (lambda (x2)
          (*-cps y y (lambda (y2)
                   (+-cps x2 y2 (lambda (x2py2)
                              (\nwlinkedidentc{sqrt-cps}{NW48t7P8-3DjkoC-1} x2py2 k))))))))
\nwindexdefn{\nwixident{distance-cps}}{distance-cps}{NW48t7P8-5trcp-1}\eatline
\nwused{\\{NW48t7P8-3l3wO2-1}}\nwidentdefs{\\{{\nwixident{distance-cps}}{distance-cps}}}\nwidentuses{\\{{\nwixident{distance}}{distance}}\\{{\nwixident{sqrt-cps}}{sqrt-cps}}}\nwindexuse{\nwixident{distance}}{distance}{NW48t7P8-5trcp-1}\nwindexuse{\nwixident{sqrt-cps}}{sqrt-cps}{NW48t7P8-5trcp-1}\nwendcode{}\nwbegindocs{4}\nwdocspar
Pour la définition des fonctions utilisées en CPS :
\nwenddocs{}\nwbegincode{5}\sublabel{NW48t7P8-3DjkoC-1}\nwmargintag{{\nwtagstyle{}\subpageref{NW48t7P8-3DjkoC-1}}}\moddef{cps function definition~{\nwtagstyle{}\subpageref{NW48t7P8-3DjkoC-1}}}\endmoddef\nwstartdeflinemarkup\nwusesondefline{\\{NW48t7P8-3l3wO2-1}}\nwenddeflinemarkup
(define (cps-prim f)
 (lambda args
  (let ((r (reverse args)))
   ((car r) (apply f
             (reverse (cdr r)))))))
(define *-cps (cps-prim *))
(define +-cps (cps-prim +))
(define \nwlinkedidentc{sqrt-cps}{NW48t7P8-3DjkoC-1} (cps-prim sqrt))
\nwindexdefn{\nwixident{cps-prim,}}{cps-prim:com}{NW48t7P8-3DjkoC-1}\nwindexdefn{\nwixident{*-cps,}}{*-cps:com}{NW48t7P8-3DjkoC-1}\nwindexdefn{\nwixident{+-cps,}}{+-cps:com}{NW48t7P8-3DjkoC-1}\nwindexdefn{\nwixident{sqrt-cps}}{sqrt-cps}{NW48t7P8-3DjkoC-1}\eatline
\nwused{\\{NW48t7P8-3l3wO2-1}}\nwidentdefs{\\{{\nwixident{*-cps,}}{*-cps:com}}\\{{\nwixident{+-cps,}}{+-cps:com}}\\{{\nwixident{cps-prim,}}{cps-prim:com}}\\{{\nwixident{sqrt-cps}}{sqrt-cps}}}\nwendcode{}\nwbegindocs{6}\nwdocspar
Testons:
\nwenddocs{}\nwbegincode{7}\sublabel{NW48t7P8-3l3wO2-1}\nwmargintag{{\nwtagstyle{}\subpageref{NW48t7P8-3l3wO2-1}}}\moddef{first-cps-test.scm~{\nwtagstyle{}\subpageref{NW48t7P8-3l3wO2-1}}}\endmoddef\nwstartdeflinemarkup\nwenddeflinemarkup
;; first-cps-test

\LA{}tools for scheme~{\nwtagstyle{}\subpageref{NW48t7P8-9KoSl-1}}\RA{}
\LA{}cps function definition~{\nwtagstyle{}\subpageref{NW48t7P8-3DjkoC-1}}\RA{}
\LA{}distance CPS~{\nwtagstyle{}\subpageref{NW48t7P8-5trcp-1}}\RA{}
\LA{}distance direct style~{\nwtagstyle{}\subpageref{NW48t7P8-1q4Jku-1}}\RA{}

(define test
  (lambda(x y)
    (\nwlinkedidentc{myprint}{NW48t7P8-9KoSl-1} "x=" x " y=" y)
    (\nwlinkedidentc{myprint}{NW48t7P8-9KoSl-1} " direct=" (\nwlinkedidentc{distance}{NW48t7P8-1q4Jku-1} x y))
    (\nwlinkedidentc{distance}{NW48t7P8-1q4Jku-1}\nwlinkedidentc{-cps}{NW48t7P8-5trcp-1} x y (lambda(e) (\nwlinkedidentc{myprint}{NW48t7P8-9KoSl-1} " cps=" e "\\n")))))

(test 3 4)
(test 0 3)
(test 3 0)
\nwnotused{first-cps-test.scm}\nwidentuses{\\{{\nwixident{distance}}{distance}}\\{{\nwixident{distance-cps}}{distance-cps}}\\{{\nwixident{myprint}}{myprint}}}\nwindexuse{\nwixident{distance}}{distance}{NW48t7P8-3l3wO2-1}\nwindexuse{\nwixident{distance-cps}}{distance-cps}{NW48t7P8-3l3wO2-1}\nwindexuse{\nwixident{myprint}}{myprint}{NW48t7P8-3l3wO2-1}\nwendcode{}\nwbegindocs{8}\nwdocspar
Ce code doit nous renvoyer ce résultat:
\nwenddocs{}\nwbegincode{9}\sublabel{NW48t7P8-3b3EwE-1}\nwmargintag{{\nwtagstyle{}\subpageref{NW48t7P8-3b3EwE-1}}}\moddef{resultat first cps test~{\nwtagstyle{}\subpageref{NW48t7P8-3b3EwE-1}}}\endmoddef\nwstartdeflinemarkup\nwenddeflinemarkup
$ gosh first-cps-test.scm
x=3 y=4 direct=5 cps=5
x=0 y=3 direct=3 cps=3
x=3 y=0 direct=3 cps=3
\nwnotused{resultat first cps test}\nwendcode{}\nwbegindocs{10}\nwdocspar

\section{du code plus intéressant}
Notre continuation un peu simpliste permet de mieux appréhender l'essentiel du problème. Voyons ce qui peut-être plus constructif et utile comme les échappements.

Voici un exemple calculant le produit des éléments d'une liste en style direct:
\nwenddocs{}\nwbegincode{11}\sublabel{NW48t7P8-23tY8l-1}\nwmargintag{{\nwtagstyle{}\subpageref{NW48t7P8-23tY8l-1}}}\moddef{list product~{\nwtagstyle{}\subpageref{NW48t7P8-23tY8l-1}}}\endmoddef\nwstartdeflinemarkup\nwprevnextdefs{\relax}{NW48t7P8-23tY8l-2}\nwenddeflinemarkup
(define \nwlinkedidentc{direct-product-1}{NW48t7P8-23tY8l-1}
  (lambda(L)
    (if (null? L)
      1
      (* (car L) (\nwlinkedidentc{direct-product-1}{NW48t7P8-23tY8l-1} (cdr L))))))
\nwindexdefn{\nwixident{direct-product-1}}{direct-product-1}{NW48t7P8-23tY8l-1}\eatline
\nwalsodefined{\\{NW48t7P8-23tY8l-2}\\{NW48t7P8-23tY8l-3}\\{NW48t7P8-23tY8l-4}}\nwnotused{list product}\nwidentdefs{\\{{\nwixident{direct-product-1}}{direct-product-1}}}\nwendcode{}\nwbegindocs{12}On peut raccourcir les calculs si $0$ est présent dans la liste:
\nwenddocs{}\nwbegincode{13}\sublabel{NW48t7P8-23tY8l-2}\nwmargintag{{\nwtagstyle{}\subpageref{NW48t7P8-23tY8l-2}}}\moddef{list product~{\nwtagstyle{}\subpageref{NW48t7P8-23tY8l-1}}}\plusendmoddef\nwstartdeflinemarkup\nwprevnextdefs{NW48t7P8-23tY8l-1}{NW48t7P8-23tY8l-3}\nwenddeflinemarkup
(define \nwlinkedidentc{direct-product-2}{NW48t7P8-23tY8l-2}
  (lambda(L)
    (cond
      ((null? L) 1)
      ((zero? (car L)) 0)
      (else (* (car L) (\nwlinkedidentc{direct-product-2}{NW48t7P8-23tY8l-2} (cdr L)))))))
\nwindexdefn{\nwixident{direct-product-2}}{direct-product-2}{NW48t7P8-23tY8l-2}\eatline
\nwidentdefs{\\{{\nwixident{direct-product-2}}{direct-product-2}}}\nwendcode{}\nwbegindocs{14}Maintenant, passons en CPS\footnote{La méthode d'obtention dans ~\cite{chazarain96}}:
\nwenddocs{}\nwbegincode{15}\sublabel{NW48t7P8-23tY8l-3}\nwmargintag{{\nwtagstyle{}\subpageref{NW48t7P8-23tY8l-3}}}\moddef{list product~{\nwtagstyle{}\subpageref{NW48t7P8-23tY8l-1}}}\plusendmoddef\nwstartdeflinemarkup\nwprevnextdefs{NW48t7P8-23tY8l-2}{NW48t7P8-23tY8l-4}\nwenddeflinemarkup
(define \nwlinkedidentc{cps-product}{NW48t7P8-23tY8l-3}
  (lambda(L k)
    (cond
      ((null? L) (k 1))
      ((zero? (car L)) (k 0))
      (else (\nwlinkedidentc{cps-product}{NW48t7P8-23tY8l-3} (cdr L) (lambda(p-cdr) (k (* p-cdr (car L))))))
      )
  ))
\nwindexdefn{\nwixident{cps-product}}{cps-product}{NW48t7P8-23tY8l-3}\eatline
\nwidentdefs{\\{{\nwixident{cps-product}}{cps-product}}}\nwendcode{}\nwbegindocs{16}\nwdocspar
L'utilisation de \texttt{cps-product} est en fait simple, il suffit d'utiliser la fonction identité:
\nwenddocs{}\nwbegincode{17}\sublabel{NW48t7P8-16ZUSX-1}\nwmargintag{{\nwtagstyle{}\subpageref{NW48t7P8-16ZUSX-1}}}\moddef{list product test~{\nwtagstyle{}\subpageref{NW48t7P8-16ZUSX-1}}}\endmoddef\nwstartdeflinemarkup\nwprevnextdefs{\relax}{NW48t7P8-16ZUSX-2}\nwenddeflinemarkup
(\nwlinkedidentc{cps-product}{NW48t7P8-23tY8l-3} '(12 6 0 35 42) (lambda(n) n))
\nwalsodefined{\\{NW48t7P8-16ZUSX-2}}\nwnotused{list product test}\nwidentuses{\\{{\nwixident{cps-product}}{cps-product}}}\nwindexuse{\nwixident{cps-product}}{cps-product}{NW48t7P8-16ZUSX-1}\nwendcode{}\nwbegindocs{18}\nwdocspar

Pour les continuations vérifiant:
\begin{equation}
   k(0) = 0
\end{equation}
on peut définir un \texttt{cps-product-0}:
\nwenddocs{}\nwbegincode{19}\sublabel{NW48t7P8-23tY8l-4}\nwmargintag{{\nwtagstyle{}\subpageref{NW48t7P8-23tY8l-4}}}\moddef{list product~{\nwtagstyle{}\subpageref{NW48t7P8-23tY8l-1}}}\plusendmoddef\nwstartdeflinemarkup\nwprevnextdefs{NW48t7P8-23tY8l-3}{\relax}\nwenddeflinemarkup
(define \nwlinkedidentc{cps-product}{NW48t7P8-23tY8l-3}\nwlinkedidentc{-0}{NW48t7P8-23tY8l-4}
  (lambda(L k)
    (cond
      ((null? L) (k 1))
      ((zero? (car L)) 0)
      (else (\nwlinkedidentc{cps-product}{NW48t7P8-23tY8l-3}\nwlinkedidentc{-0}{NW48t7P8-23tY8l-4} (cdr L) (lambda(p-cdr) (k (* p-cdr (car L))))))
      )
  ))
\nwindexdefn{\nwixident{cps-product-0}}{cps-product-0}{NW48t7P8-23tY8l-4}\eatline
\nwidentdefs{\\{{\nwixident{cps-product-0}}{cps-product-0}}}\nwidentuses{\\{{\nwixident{cps-product}}{cps-product}}}\nwindexuse{\nwixident{cps-product}}{cps-product}{NW48t7P8-23tY8l-4}\nwendcode{}\nwbegindocs{20}\nwdocspar
Ici, l'avantage de la continuation est plus évident. Si nous voulons obtenir la somme de carré des éléments de la liste, on réécrit le précédent exemple ainsi :
\nwenddocs{}\nwbegincode{21}\sublabel{NW48t7P8-16ZUSX-2}\nwmargintag{{\nwtagstyle{}\subpageref{NW48t7P8-16ZUSX-2}}}\moddef{list product test~{\nwtagstyle{}\subpageref{NW48t7P8-16ZUSX-1}}}\plusendmoddef\nwstartdeflinemarkup\nwprevnextdefs{NW48t7P8-16ZUSX-1}{\relax}\nwenddeflinemarkup
;; avec les carré :
(\nwlinkedidentc{cps-product}{NW48t7P8-23tY8l-3} '(12 6 0 35 42) (lambda(n) (* n n)))
\nwidentuses{\\{{\nwixident{cps-product}}{cps-product}}}\nwindexuse{\nwixident{cps-product}}{cps-product}{NW48t7P8-16ZUSX-2}\nwendcode{}\nwbegindocs{22}\nwdocspar

Bien qu'il soit dit en de nombreux ouvrages que la CPS pouvait rendre le code très difficile à écrire, lire et entretenir, on voit bien qu'ici on obtient une ouverture intéressante sur des possibilités inattendues comme une gestion d'erreur.
\part{annexes}
\section{fonctions utiles}

\subsection{affichage console}
Voici un \texttt{display} plus \emph{fun}:
\nwenddocs{}\nwbegincode{23}\sublabel{NW48t7P8-9KoSl-1}\nwmargintag{{\nwtagstyle{}\subpageref{NW48t7P8-9KoSl-1}}}\moddef{tools for scheme~{\nwtagstyle{}\subpageref{NW48t7P8-9KoSl-1}}}\endmoddef\nwstartdeflinemarkup\nwusesondefline{\\{NW48t7P8-3l3wO2-1}}\nwenddeflinemarkup
(define (\nwlinkedidentc{myprint}{NW48t7P8-9KoSl-1} . l)
  (for-each (lambda(e) (display e)) l))
\nwindexdefn{\nwixident{myprint}}{myprint}{NW48t7P8-9KoSl-1}\eatline
\nwused{\\{NW48t7P8-3l3wO2-1}}\nwidentdefs{\\{{\nwixident{myprint}}{myprint}}}\nwendcode{}

\nwixlogsorted{c}{{cps function definition}{NW48t7P8-3DjkoC-1}{\nwixd{NW48t7P8-3DjkoC-1}\nwixu{NW48t7P8-3l3wO2-1}}}%
\nwixlogsorted{c}{{distance CPS}{NW48t7P8-5trcp-1}{\nwixd{NW48t7P8-5trcp-1}\nwixu{NW48t7P8-3l3wO2-1}}}%
\nwixlogsorted{c}{{distance direct style}{NW48t7P8-1q4Jku-1}{\nwixd{NW48t7P8-1q4Jku-1}\nwixu{NW48t7P8-3l3wO2-1}}}%
\nwixlogsorted{c}{{first-cps-test.scm}{NW48t7P8-3l3wO2-1}{\nwixd{NW48t7P8-3l3wO2-1}}}%
\nwixlogsorted{c}{{list product}{NW48t7P8-23tY8l-1}{\nwixd{NW48t7P8-23tY8l-1}\nwixd{NW48t7P8-23tY8l-2}\nwixd{NW48t7P8-23tY8l-3}\nwixd{NW48t7P8-23tY8l-4}}}%
\nwixlogsorted{c}{{list product test}{NW48t7P8-16ZUSX-1}{\nwixd{NW48t7P8-16ZUSX-1}\nwixd{NW48t7P8-16ZUSX-2}}}%
\nwixlogsorted{c}{{resultat first cps test}{NW48t7P8-3b3EwE-1}{\nwixd{NW48t7P8-3b3EwE-1}}}%
\nwixlogsorted{c}{{tools for scheme}{NW48t7P8-9KoSl-1}{\nwixu{NW48t7P8-3l3wO2-1}\nwixd{NW48t7P8-9KoSl-1}}}%
\nwixlogsorted{i}{{\nwixident{*-cps,}}{*-cps:com}}%
\nwixlogsorted{i}{{\nwixident{+-cps,}}{+-cps:com}}%
\nwixlogsorted{i}{{\nwixident{cps-prim,}}{cps-prim:com}}%
\nwixlogsorted{i}{{\nwixident{cps-product}}{cps-product}}%
\nwixlogsorted{i}{{\nwixident{cps-product-0}}{cps-product-0}}%
\nwixlogsorted{i}{{\nwixident{direct-product-1}}{direct-product-1}}%
\nwixlogsorted{i}{{\nwixident{direct-product-2}}{direct-product-2}}%
\nwixlogsorted{i}{{\nwixident{distance}}{distance}}%
\nwixlogsorted{i}{{\nwixident{distance-cps}}{distance-cps}}%
\nwixlogsorted{i}{{\nwixident{myprint}}{myprint}}%
\nwixlogsorted{i}{{\nwixident{sqrt-cps}}{sqrt-cps}}%
\nwbegindocs{24}\nwdocspar

\part{tables et index}
\section{table des extraits de code}

\begin{multicols}{2}
\nowebchunks
\end{multicols}

\section{index des symboles}

\begin{multicols}{2}
\nowebindex
\end{multicols}

\listofdefinitions

\section{bibliographie}

\bibliographystyle{plainnat}
% \bibliographystyle{achicago}
\bibliography{biblio/computing}

\end{document}
\nwenddocs{}
